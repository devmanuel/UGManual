\section{Outline of the UG Program at IIT Goa}
	
Currently, the only \acrshort{ug} degree program offered by \acrfull{iitgoa} is \acrfull{btech}. The \glspl{academicprogram} at \acrshort{iitgoa} started in 2016 are Mechanical Engineering, Electrical engineering, and Computer science while Mathematics and Computing was introduced in 2019. The objectives of the undergraduate program at \acrshort{iitgoa} are stated below.
	
\begin{itemize}[leftmargin=15mm]
	\item To provide the highest level of education in technology and science, and to produce competent, creative, and imaginative professionals.
	\item To promote a spirit of free and objective enquiry, and development of knowledge in different disciplines. 
	\item To produce highly skilled technologists with entrepreneurial skills having team spirit and leadership qualities. 
	\item To increase \gls{student} participation in nation building through technology development that is sensitive to local needs.
\end{itemize}
	 
The \acrshort{ug} program at \acrshort{iitgoa} consists of courses in basic sciences, humanities and social sciences, engineering, technology, and other related topics. The sequence of studies broadly consists of four stages.
	
\begin{itemize}[leftmargin=15mm]
	\item The first stage is an introduction to the sciences, humanities, and technical arts (such as workshops and engineering graphics). This is common and mandatory for all academic programs. 
	\item The second stage is the study of engineering courses that emphasizes on developing broad-based knowledge in core as well as interdisciplinary areas which enables the \glspl{student} to appreciate the links between science, engineering, and humanities. 
	\item In the third stage, the \glspl{student} are exposed to subjects in their chosen areas of study which develop in them the ability for physical/ analytical modeling, design and development. 
	\item During the final stage, an engineering student has the opportunity to study problems of integrated design with an awareness of size, performance, optimization and cost. Here they may work independently on a B.Tech. Project or engage in an industrial internship (see \Cref{lab:BTPandInternship}) where they will receive valuable hands-on experience and enhanced potential job opportunities.
\end{itemize}

In parallel with the last two stages, the student is introduced to the social and economic objectives of the era and to the interaction between man, machine, and nature. This is sought to be achieved through elective courses in humanities and social sciences as well as through practical training, fieldwork, works-visits, and seminars. 