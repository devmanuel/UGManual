\section{Registration \label{lab:Registration} }

Each admitted \gls{student} is required to register before commencement of a semester in order to enroll for that particular semester. The registration process consists of administrative registration and academic registration.

\subsection{Administrative Registration}

Administrative registration deals with matters related to fee payment. \Glspl{student} must comply to the following to complete the administrative registration.

\begin{enumerate}[leftmargin=15mm]
    \item Payment of all fees as applicable and clearance of all outstanding dues if any.
    \item Issue or renewal of identity cards if required as per the \gls{institute} rules.
    \item Any other process stipulated by the \gls{institute}.
\end{enumerate}

\subsection{Academic Registration}

Academic registration involves the selection of courses consistent with the specified credit requirements and is subject to the rules described below. Registration is done at the beginning of each semester on the prescribed dates announced in the academic calendar and it is mandatory for every student until they complete their academic program. The \gls{student} may be considered de-registered from the program if they fail to register during the specified window.

On joining the \gls{institute}, each \gls{student} is assigned to a Faculty Adviser (\Cref{lab:Faculty Advisor} and the student can register for courses only with the approval from their Faculty Advisers. \Glspl{student} should contact their respective Faculty Advisers for consultation on selection of courses and registration.

\acrshort{iitgoa} has an online registration system accessible through the \gls{institute} website. The registration instructions are shared with registration email and \gls{institute} intranet.

\paragraph{Registration for the First Two Semesters \label{lab:First Year Registration}}

In each of the first two semesters of the program, a student is required to register for all the courses listed in the curriculum for the semester. In addition, students who are identified as academically weak at the end of the first semester may be prescribed a specially worked out load in consultation with the Faculty Advisor (\Cref{lab:Faculty Advisor}).

\paragraph{Registration for Third and Subsequent semesters}

From third semester onwards, a student is required to register for the number of credits as prescribed by the curriculum of their program taking into account the provisions in the academic progression policy. The faculty advisor (\Cref{lab:Faculty Advisor}) may recommend a reduced load for \glspl{student} with backlogs after considering the regulations as decided by the \acrshort{sugc} in the academic progression policy (see \Cref{lab:Policy on Academic Progression}).

\paragraph{Registration for Summer Term} Only in case the number of \glspl{student} with backlog (see \Cref{lab:Backlog Course}) in a course is greater than or equal to five, the concerned program may offer this course during summer vacation subject to availability or willingness of the \gls{faculty}.

\begin{enumerate}[leftmargin=15mm]
    \item Summer courses with less number of \glspl{student} are discouraged. Such a proposal should come from the \acrshort{pugc} and be approved by the \acrshort{deanap}.
    \item \Glspl{student} can register for a maximum of \textit{four} courses during summer program on payment of the prescribed registration fees.
    \item The course instructor will monitor the attendance of the \glspl{student} registered and he/she may award grade 'X' (see \Cref{lab:Grading Policy}) as per rule for poor attendance.
\end{enumerate}

   The summer course facility provides the opportunity to the \glspl{student} to clear their backlogs, by re-doing courses with adequate rigor, provided it is offered. \Glspl{student} are also not permitted to re-register for courses, which they have already obtained a passing grade. The registration, examination etc. will be as per academic calendar (see \Cref{lab:Academic Calendar}) and the evaluation/ grading will be done in the similar way as is done for normal semester courses. 

\subsection{Rules Regarding Academic Registration}

\paragraph{General Rule}

\begin{enumerate}[leftmargin=15mm]
    \item A registration is considered valid only if there is no timetable conflict between the courses for which the \gls{student} has registered. 
    \item The registration of a \gls{student} in a course may be cancelled at any stage by the Academic office, if it is found that they do not meet the prerequisites for the course, or if there is a clash in the \gls{student}'s time table preventing her/him from attending the course or if it is found that s/he is not eligible to register for that course for any other reason.
    \item Registrations must be completed on or before the prescribed last date for registration. Late registration may be permitted only for valid reasons and on payment of a late registration fee as prescribed from time to time. 
    \item If a \gls{student} does not register for a regular semester (Autumn or Spring) without prior permission from \acrshort{sugc} or fails to register during any semester within the prescribed timeline, the \gls{student} will be de-registered from the program and considered terminated.
\end{enumerate}

\paragraph{Adding and Dropping of Courses} 

\begin{enumerate}[leftmargin=15mm, resume]
    \item From third semester onwards, the \glspl{student} may choose to drop one or two courses out of the registered one(s), provided the minimum credit requirement is fulfilled. 
    \item The \glspl{student}, however, will not be permitted to drop backlog course/s for which they have registered. Course/s dropped by a \gls{student} may be taken during the summer term (if offered) or during a subsequent semester. 
    \item In all the cases, course adjustments/dropping of courses must be done before the last date for Course add and drop, as announced by the \gls{institute} in its academic calendar. 
    \item In exceptional cases, \acrshort{sugc} may allow adding/dropping of courses beyond deadline upon recommendation by the faculty adviser and the \acrshort{pugc}. This is typically reserved for medical emergencies. 
    \item Adding of courses is not permitted in the summer term. However, \glspl{student} may drop a course up to two weeks prior to the last day of classes for summer course.
\end{enumerate}

\paragraph{Repeating a Course} A \gls{student} is required to repeat a course completely under the following situations: 

\begin{enumerate}[leftmargin=15mm, resume]
    \item When s/he gets a failing grade in a course. 
    \item When a \gls{student}, who gets a grade ‘IN' (see \Cref{lab:Grading Policy}) and fails to apply or does not appear for a re-examination (giving valid/ medical reasons for absence at the end-semester examination, the grade 'F' (see \Cref{lab:Grading Policy}) is awarded automatically to such course).
\end{enumerate}

\paragraph{Course Substitution} The substitution of one course where the \gls{student} has failed by another course in a later semester is governed by following rules.

\begin{enumerate}[leftmargin=15mm, resume]
    \item For an institute core or program core course (compulsory courses), no substitution is allowed and the same course must be repeated. 
    \item For an open elective course, substitution by another open elective is the only possibility. 
    \item Program electives may be substituted by another program elective course.
    \item Substitution of courses with previously overloaded courses is governed by the academic progression rules.
\end{enumerate}

\acrshort{sugc} is the approving authority for the course substitution.

\subsection{Special Features in Registration}

\paragraph{Audit Course} Auditing of courses by the Undergraduate \glspl{student} during regular semesters and summer term is permitted under the following conditions

\begin{enumerate}[leftmargin=15mm]
    \item \Glspl{student} with CPI 7.5 and above will be permitted to ‘Audit' the course. However, this would be restricted to maximum of 2 courses, irrespective of passed or failed during the entire period of the program. 
    \item The audited course will not carry any credits. The course done by auditing will not be considered for the purposes of calculation of SPI/CPI but will be reflected in the Semester Grade Report as Audit Course. 
    \item Prior permission of the Instructor is required. 
    \item The grade ‘L' (see \Cref{lab:Grading Policy}) would be awarded by the instructor if the attendance is satisfactory (minimum 80\%) and requirements set out by the instructor are met. \Glspl{student} will be expected to complete the in-semester assessments. If the attendance and performance is \textit{not} satisfactory grade ‘FL’ (see \Cref{lab:Grading Policy}) would be awarded and the course will not appear at all in the grade card. 
    \item \Glspl{student} can audit a course provided the course is offered and the timetable slot permits. 
\end{enumerate}

\paragraph{Self-Study} Self-study mode of crediting a course is intended to provide a \gls{student} with a fail grade in the given course an opportunity to complete the course credit requirements by the end of fourth year of \acrshort{btech} program. The self-study mode will allow \glspl{student} to avoid time-table clash while registering for backlog courses (see Section 7.3.\ref{lab:Backlog Course}). \Gls{student} may request for self-study courses by taking necessary approval from \acrshort{pugc} and \acrshort{sugc}. Following are the modalities of the self-study.

\begin{enumerate} [leftmargin=15mm]
    \item Self-study courses can be requested only during a regular semester and only for courses which are typically offered in that semester. These courses are subject to the availability/willingness of concerned \glspl{faculty}.
    \item A maximum of \textit{three} courses can be taken in self study mode, only from the \textit{sixth} semester onwards.
    \item An application for self-study has to be made to the \acrshort{pugc}, along with the permission from the instructor of the course well before the last date for adding/ dropping courses. 
    \item The \acrshort{sugc} will consider each application for self-study on its merits and will restrict the total number of such courses to only three during the entire program of a \gls{student}. \Glspl{student} who have not attempted to clear the course in a regular manner may not be considered eligible for this mode of crediting. 
    \item The registration, examination etc. will be as per academic calendar. The course should be completed and the grade obtained in the same semester in which the registration is done without any carryover from one semester to another. 
    \item The evaluation for self-study will be done in the similar way as is done for normal semester course. Academic standards must be rigorously maintained in the self-study mode. The instructor must supervise the \gls{student} from time to time apart from an examination at the end of the course.
    \item The credits for a self-study course are the same as those for the regular course and will be taken into account while calculating the total credits in a semester which should not exceed the normal load. 
\end{enumerate}

\paragraph{Guided Study} The capable \glspl{student} will be provided option of Guided study to acquire proficiency in an area of their choice, through doing courses outside their curriculum by a self-study like mode. This option is also subject to the availability and willingness of the instructor offering the course. Such an option will be available to \glspl{student} with a CPI of 8.5 or more to the extent of a maximum of one course per semester in the third and subsequent years, on the recommendation of the faculty adviser (see \Cref{lab:Faculty Advisor})/ project guide, and approval of \acrshort{pugc}. This option is called ‘Guided Study' to distinguish it from the existing self-study mode, which is largely meant for \glspl{student} to clear backlogs in the curriculum. Further, even in respect of these courses, the rules that govern overloading apply strictly. The Guided study option may be used, on a case-by-case basis with prior approval of \acrshort{pugc}. The registration, examination etc. will be similar to a self-study course. The evaluation for guided study will be done in the similar way as is done for normal courses. This option will be made available in the following special situations also, provided CPI requirement is fulfilled: 

\begin{enumerate} [leftmargin=15mm]
    \item \Glspl{student} who miss the ‘Introduction to Profession Course' due to a change of branch obtained at the end of first year. 
    \item \Glspl{student} dropping an entire semester due to medical reasons. 
\end{enumerate}