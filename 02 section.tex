\section{The Academic Administration}

\subsection{Academic Office (AO)}

The academic office (AO) is headed by the Dean of Academic Programs (Dean (AP)) or the Faculty-in-charge and it executes the decisions taken by the senate. In consultation with the standing committees formed by the senate, AO is responsible for the following tasks.

\begin{enumerate}
	\item Coordination of admissions to various UG programs, processing and maintenance of all records related to admissions
	\item Coordination of academic activities such as course registrations, preparation of academic calendar, scheduling of courses, examinations and maintenance of all records related to these activities as well the processing of cases of academic malpractices.
	\item Implementation of academic progression and termination.
	\item Implementation of leave rules.
	\item Coordination of curriculum implementation, introduction of new courses, revisions to the existing courses, revisions to manuals duly approved by the senate.
	\item Conduct of Convocation, award of prizes and issuing transcripts and degree certificates.
	\item Dissemination of information pertaining to all academic matters to students and faculty.
	\item Issuing necessary communications including circulars, orders. 
	\item Acting as a channel of communication between students, faculty members, schools, senate standing committees, and the senate.
\end{enumerate}

\subsection{Student Affairs Office (SAO)}

The student affairs office is headed by Dean of Student Affairs (Dean (SA)) the Faculty-in-Charge. It is primarily responsible for ensuring the well-being of students on campus through planning, coordinating and executing pertinent activities. In consultation with various faculty advisors and wardens, SAO undertakes following responsibilities:

\begin{enumerate}
	\item Coordinating the extra and co-curricular activities for students in three broad domains: Technical, Cultural and Sports (in close consultation with respective Faculty Advisors).
	\item Ensuring the maintenance of hostel premises including the quality of catering services in the hostel mess (with the help of mess and hostel council including student members).
	\item Coordinating the allotment of hostel rooms for all the eligible student (through hall office in consultation with wardens).
	\item Coordination of activities such as collection of fees and processing of scholarships.
	\item Ensuring the observation of hostel rules and code of conduct by the residents and issuing necessary penalty in case of any deviation (through hall office in consultation with wardens).
	\item Ensuring the mental well-being of students through preventive wellness activities as well as counselling (through Wellness and Counselling Cell).
	\item Conducting the yearly elections for the Students’ Council.
	\item Coordinating the Alumni Cell activities (through members of the alumni cell).
\end{enumerate}

\subsection{Senate Undergraduate Committee (SUGC)}

This is a standing committee formed by the senate to oversee and monitor UG programs across the institute. It makes recommendations to the senate on all academic issues including policy matters as well as specific problems related to registered UG students and UG programs. Its constituents are representatives from various Academic Units (AUs), last SUGC chairperson (ex-officio), and four student representatives elected/nominated by the student council. The student members do not participate when the cases of academic evaluations of students are considered. Although the student members' opinion might be sought before making any decision, student members have no voting rights. The SUGC chairperson is elected by the constituent members. 

SUGC is responsible for the following tasks in consultation of AUs and PUGCs.

\begin{enumerate}
	\item Making recommendations to the AO in all matters pertaining to academics, including the introduction of new courses, credits allotted to the courses, approval for their contents, and evaluation policies.
	\item Recommending modifications, as appropriate, for courses that are already approved by the senate.
	\item Implementation of various aspects of the core curriculum and modifications if any.
	\item Making recommendations in cases of academic warning, probation, termination, and appeal against termination.
	\item Overseeing the processes of amending and updating the UG manual as and when required. 
\end{enumerate}

\subsection{Program Undergraduate Committee (PUGC)}

Each program has a Program Undergraduate Committee (PUGC) to assist SUGC. Preferably, a PUGC member from respective AU is nominated as the member of SUGC. The Program Undergraduate Committee (PUGC) that consists of the Program Coordinator (PC), a convener nominated by the PC in consultation with the faculty of the program, one to three faculty members chosen by the program, and at least two nominated undergraduate students. The tenure of a faculty member in PUGC is two years, one third of them retiring each year and student member’s tenure is one year. The PUGC has the following functions.

\begin{enumerate}
	\item Advises the students about the curriculum of the program. 
	\item Advises them about academic opportunities. 
	\item Monitors the progress of academically weak students. 
	\item Handles any problem faced by students in their academic programs.
	\item Assist SUGC on all academic matters whenever it is required.
\end{enumerate}

\subsection{The Faculty Advisor}

On joining the Institute, each student is assigned a Faculty Adviser (FA). The FA guides the students to complete their courses of study for the required degree. For effective utilization of the opportunities for additional academic accomplishments, the planning of an individual’s academic journey needs careful consideration, and hence constant consultation with the FA is imperative. FA approves the registration of the students at the beginning of each semester and guides the students about the rules and regulations governing their courses of study. The FA should be available to discuss with the students their academic performance during the previous semester and help the student decide on the courses for which they can register.