\section{The Academic Administration}

\subsection{Academic Office}

The academic office (\acrshort{ao}) is headed by the \acrfull{deanap} or the Faculty-in-charge and it executes the decisions taken by the senate. In consultation with the standing committees formed by the \Gls{senate}, \acrshort{ao} is responsible for the following tasks.

\begin{itemize}[leftmargin=15mm]
	\item Coordination of admissions to various \acrshort{ug} programs, processing and maintenance of all records related to admissions.
	\item Coordination of academic activities such as course registrations, preparation of academic calendar, scheduling of courses, examinations and maintenance of all records related to these activities as well the processing of cases of academic malpractices.
	\item Implementation of academic progression and termination.
	\item Implementation of leave rules.
	\item Coordination of curriculum implementation, introduction of new courses, revisions to the existing courses, revisions to manuals duly approved by the \gls{senate}.
	\item Conduct of Convocation, award of prizes and issuing transcripts and degree certificates.
	\item Dissemination of information pertaining to all academic matters to \glspl{student} and faculty.
	\item Issuing necessary communications including circulars, orders. 
	\item Acting as a channel of communication between \glspl{student}, \glspl{faculty}, \gls{school}, senate standing committees, and the \gls{senate}.
\end{itemize}

\subsection{Student Affairs Office}

The student affairs office (\acrshort{sao}) is headed by \acrfull{deansa} or the Faculty-in-Charge. It is primarily responsible for ensuring the well-being of students on campus through planning, coordinating and executing pertinent activities. In consultation with various faculty advisors and wardens, \acrshort{sao} undertakes following responsibilities:

\begin{itemize}[leftmargin=15mm]
	\item Coordinating the extra and co-curricular activities for \glspl{student} in three broad domains: Technical, Cultural and Sports (in close consultation with respective Faculty Advisors).
	\item Ensuring the maintenance of hostel premises including the quality of catering services in the hostel mess (with the help of mess and hostel council including student members).
	\item Coordinating the allotment of hostel rooms for all the eligible \gls{student} (through hall office in consultation with wardens).
	\item Coordination of activities such as collection of fees and processing of scholarships.
	\item Ensuring the observation of hostel rules and code of conduct by the residents and issuing necessary penalty in case of any deviation (through hall office in consultation with wardens).
	\item Ensuring the mental well-being of \glspl{student} through preventive wellness activities as well as counselling (through Wellness and Counselling Cell).
	\item Conducting the yearly elections for the Students’ Council.
	\item Coordinating the Alumni Cell activities (through members of the alumni cell).
\end{itemize}

\subsection{Senate Undergraduate Committee}

\acrfull{sugc} is a standing committee formed by the \gls{senate} to oversee and monitor \acrshort{ug} programs across the \gls{institute}. It makes recommendations to the \gls{senate} on all academic issues including policy matters as well as specific problems related to registered \acrshort{ug} students and \acrshort{ug} programs. Its constituents are representatives from various \acrfullpl{au}, last \acrshort{sugc} chairperson (ex-officio), and four student representatives elected/nominated by the student council. The student members do not participate when the cases of academic evaluations of students are considered. Although the student members' opinion might be sought before making any decision, student members have \textit{no} voting rights. The \acrshort{sugc} chairperson is elected by the constituent members. 

\acrshort{sugc} is responsible for the following tasks in consultation of \acrshortpl{au} and \acrshortpl{pugc}.

\begin{itemize}[leftmargin=15mm]
	\item Making recommendations to the \acrshort{ao} in  all matters pertaining to academics, including the introduction of new courses, credits allotted to the courses, approval for their contents, and evaluation policies.
	\item Recommending modifications, as appropriate, for courses that are already approved by the \gls{senate}.
	\item Implementation of various aspects of the core curriculum and modifications if any.
	\item Making recommendations in cases of academic warning, probation, termination, and appeal against termination.
	\item Overseeing the processes of amending and updating the \acrshort{ug} manual as and when required. 
\end{itemize}

\subsection{Program Undergraduate Committee \label{lab:PUGC}}

Each \gls{program} has a \acrfull{pugc} to assist \acrshort{sugc}. Preferably, a \acrshort{pugc} member from respective \acrshort{au} is nominated as the member of \acrshort{sugc}. The \acrfull{pugc} that consists of the \Gls{pc}, a convener nominated by the \acrshort{pc} in consultation with the \gls{faculty} of the \gls{program}, one to three \glspl{faculty} chosen by the \gls{program}, and at least two nominated undergraduate \glspl{student}. The tenure of a \gls{faculty} in \acrshort{pugc} is two years, one third of them retiring each year and student member’s tenure is one year. The \acrshort{pugc} has the following functions.

\begin{itemize}[leftmargin=15mm]
	\item Advises the \glspl{student} about the curriculum (see \Cref{lab:Curriculum}) of the \gls{program}. 
	\item Advises the \glspl{student} about academic opportunities. 
	\item Monitors the progress of academically weak \glspl{student}. 
	\item Handles any problem faced by \glspl{student} in their academic programs.
	\item Assist \acrshort{sugc} on all academic matters whenever it is required.
\end{itemize}

\subsection{The Faculty Advisor \label{lab:Faculty Advisor}}

On joining the \gls{institute}, each \gls{student} is assigned a Faculty Adviser (FA). The FA guides the \glspl{student} to complete their courses of study for the required degree. For effective utilization of the opportunities for additional academic accomplishments, the planning of an individual’s academic journey needs careful consideration, and hence constant consultation with the FA is imperative. FA approves the registration of the \glspl{student} at the beginning of each semester (see \Cref{lab:Semester}) and guides the \glspl{student} about the rules and regulations governing their courses of study. The FA should be available to discuss with the \glspl{student} their academic performance during the previous semester and help the \gls{student} decide on the courses for which they can register.