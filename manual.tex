%\author{Academic Office}

\documentclass[a4paper, 12pt]{article}
\usepackage{array, booktabs, multirow, makecell}
\usepackage[sfdefault]{atkinson}

\setlength{\parindent}{0pt}

\title{{\Huge Undergraduate Manual }}

\begin{document}
\maketitle
\pagebreak

%%ADD Preamble

%%ADD Glossary

\section{Outline of the UG Program at IIT Goa}
	
	Currently, the only UG degree program offered by IIT Goa is Bachelor of Technology (B.Tech.). The academic programs at IIT Goa started in 2016 are Mechanical Engineering, Electrical engineering, and Computer science while Mathematics and Computing was introduced in 2019. The objectives of the undergraduate program at IIT Goa are stated below.
	
	\begin{enumerate}
	\item To provide the highest level of education in technology and science, and to produce competent, creative, and imaginative professionals.
	\item To promote a spirit of free and objective enquiry, and development of knowledge in different disciplines. 
	\item To produce highly skilled technologists with entrepreneurial skills having team spirit and leadership qualities. 
	\item To increase student participation in nation building through technology development that is sensitive to local needs.
	\end{enumerate}
	 
	The UG program at IIT Goa consists of courses in basic sciences, humanities and social sciences, engineering, technology, and other related topics. The sequence of studies broadly consists of four stages.
	\begin{enumerate}
		 
	\item The first stage is an introduction to the sciences, humanities, and technical arts (such as workshops and engineering graphics). This is common and mandatory for all academic programs. 
	\item The second stage is the study of engineering courses that emphasizes on developing broad-based knowledge in core as well as interdisciplinary areas which enables the students to appreciate the links between science, engineering, and humanities. 
	\item In the third stage, the students are exposed to subjects in their chosen areas of study which develop in them the ability for physical/ analytical modeling, design and development. 
	\item During the final stage, an engineering student has the opportunity to study problems of integrated design with an awareness of size, performance, optimization and cost. Here they may work independently on a B.Tech. Project or engage in an industrial internship where they will receive valuable hands-on experience and enhanced potential job opportunities.
\end{enumerate}

	In parallel with the last two stages, the student is introduced to the social and economic objectives of the era and to the interaction between man, machine, and nature. This is sought to be achieved through elective courses in humanities and social sciences as well as through practical training, fieldwork, works-visits, and seminars. 


\section{The Academic Administration}

\subsection{Academic Office (AO)}

The academic office (AO) is headed by the Dean of Academic Programs \textit{(Dean (AP)}) or the Faculty-in-charge and it executes the decisions taken by the senate. In consultation with the standing committees formed by the senate, AO is responsible for the following tasks.

\begin{enumerate}
	\item Coordination of admissions to various UG programs, processing and maintenance of all records related to admissions
	\item Coordination of academic activities such as course registrations, preparation of academic calendar, scheduling of courses, examinations and maintenance of all records related to these activities as well the processing of cases of academic malpractices.
	\item Implementation of academic progression and termination.
	\item Implementation of leave rules.
	\item Coordination of curriculum implementation, introduction of new courses, revisions to the existing courses, revisions to manuals duly approved by the senate.
	\item Conduct of Convocation, award of prizes and issuing transcripts and degree certificates.
	\item Dissemination of information pertaining to all academic matters to students and faculty.
	\item Issuing necessary communications including circulars, orders. 
	\item Acting as a channel of communication between students, faculty members, schools, senate standing committees, and the senate.
\end{enumerate}

\subsection{Student Affairs Office (SAO)}

The student affairs office is headed by Dean of Student Affairs (Dean (SA)) the Faculty-in-Charge. It is primarily responsible for ensuring the well-being of students on campus through planning, coordinating and executing pertinent activities. In consultation with various faculty advisors and wardens, SAO undertakes following responsibilities:

\begin{enumerate}
	\item Coordinating the extra and co-curricular activities for students in three broad domains: Technical, Cultural and Sports (in close consultation with respective Faculty Advisors).
	\item Ensuring the maintenance of hostel premises including the quality of catering services in the hostel mess (with the help of mess and hostel council including student members).
	\item Coordinating the allotment of hostel rooms for all the eligible student (through hall office in consultation with wardens).
	\item Coordination of activities such as collection of fees and processing of scholarships.
	\item Ensuring the observation of hostel rules and code of conduct by the residents and issuing necessary penalty in case of any deviation (through hall office in consultation with wardens).
	\item Ensuring the mental well-being of students through preventive wellness activities as well as counselling (through Wellness and Counselling Cell).
	\item Conducting the yearly elections for the Students’ Council.
	\item Coordinating the Alumni Cell activities (through members of the alumni cell).
\end{enumerate}

\subsection{Senate Undergraduate Committee (SUGC)}

This is a standing committee formed by the senate to oversee and monitor UG programs across the institute. It makes recommendations to the senate on all academic issues including policy matters as well as specific problems related to registered UG students and UG programs. Its constituents are representatives from various Academic Units (AUs), last SUGC chairperson (ex-officio), and four student representatives elected/nominated by the student council. The student members do not participate when the cases of academic evaluations of students are considered. Although the student members' opinion might be sought before making any decision, student members have no voting rights. The SUGC chairperson is elected by the constituent members. 

SUGC is responsible for the following tasks in consultation of AUs and PUGCs.

\begin{enumerate}
	\item Making recommendations to the AO in all matters pertaining to academics, including the introduction of new courses, credits allotted to the courses, approval for their contents, and evaluation policies.
	\item Recommending modifications, as appropriate, for courses that are already approved by the senate.
	\item Implementation of various aspects of the core curriculum and modifications if any.
	\item Making recommendations in cases of academic warning, probation, termination, and appeal against termination.
	\item Overseeing the processes of amending and updating the UG manual as and when required. 
\end{enumerate}

\subsection{Program Undergraduate Committee (PUGC)}

Each program has a Program Undergraduate Committee (PUGC) to assist SUGC. Preferably, a PUGC member from respective AU is nominated as the member of SUGC. The Program Undergraduate Committee (PUGC) that consists of the Program Coordinator (PC), a convener nominated by the PC in consultation with the faculty of the program, one to three faculty members chosen by the program, and at least two nominated undergraduate students. The tenure of a faculty member in PUGC is two years, one third of them retiring each year and student member’s tenure is one year. The PUGC has the following functions.

\begin{enumerate}
	\item Advises the students about the curriculum of the program. 
	\item Advises them about academic opportunities. 
	\item Monitors the progress of academically weak students. 
	\item Handles any problem faced by students in their academic programs.
	\item Assist SUGC on all academic matters whenever it is required.
\end{enumerate}

\subsection{The Faculty Advisor}

On joining the Institute, each student is assigned a Faculty Adviser (FA). The FA guides the students to complete their courses of study for the required degree. For effective utilization of the opportunities for additional academic accomplishments, the planning of an individual’s academic journey needs careful consideration, and hence constant consultation with the FA is imperative. FA approves the registration of the students at the beginning of each semester and guides the students about the rules and regulations governing their courses of study. The FA should be available to discuss with the students their academic performance during the previous semester and help the student decide on the courses for which they can register.

\section{Academic Sessions, Credit system, and Programs of Study}

\subsection{Semesters}

The academic session normally runs from the end of July in one year to the middle of July in the next year. It is typically divided into three parts described below.

\begin{enumerate}
	\item Autumn Semester: From the fourth week of July to the last week of November.
	\item Spring Semester: From the last week of December to the last week of April.
	\item Summer Term (not a regular semester): From the middle of May to the middle of July.
\end{enumerate}

Each of the two regular semesters consists of about seventeen to nineteen weeks. About seven to nine working days of each semester are used for the end-semester examination and one week during the semester is utilized for the mid-semester examination. The equivalent of fourteen weeks is devoted to teaching, which excludes all holidays and days spent on both the examinations, in each semester. The summer term consists of eight teaching weeks, not including holidays and examinations days.

\subsection{Academic Calendar}

The dates of all academic activities including those of registration, late registration, last date of document submission, first and the last days of classes, dates to add or drop courses, course feedback collection, examinations, deadline for final grade submission, mid-semester recess, and vacation are published in the academic calendar every year by the academic office.

\subsection{Credit System}

The students admitted to the UG program are expected to complete eight semesters of academic work. IIT Goa employs an academic credit system that is a standard used to quantify the efforts taken by the student towards their academic targets. Each course is assigned a certain number of credits based on the number of engagement hours of instruction. Their credits are calculated by means of the lecture hours per week (L), tutorial hours per week (T) and practical hours per week (P). Each program specifies the total number of credits required for graduation as described in a later section. The courses students can take are categorized as either standard courses or modular courses.

\subsubsection{Standard courses}

These courses are either full semester or half-semester long. The calculation of credits for standard courses depends on whether the course involves lecture hours. For a full semester course, when the course involves lectures, the credit is computed as shown below.


	\[Credits=L+T+floor(\frac{P}{2})\]


If the course contains no lecture hours, the credits are calculated using the formula below.


	\[Credits=L+T+ceiling(\frac{P}{2})\]


The credit of a half-semester course is half of that of a full semester course with similar L-T-P parameters. The table below exemplifies the above procedure for various combinations of L-T-P hours.

\begin{table}[!h]
	\centering
	
\begin{tabular}{|c|c|c|c|c|c|}
%Add the first table
\end{tabular}

	\caption{ Illustration of course credits for different L-T-P parameters}
	
\end{table}

\subsubsection{Modular Courses}

A modular course is one that has a shorter duration than a half semester courses and is flexible in its schedule compared to a standard course. Unlike a standard course, a modular course can start any time during the semester and its scheduling could be of a different form. For instance, a modular course could consist of engagement hours totaling to 14 hours in a span of 3 weeks or over several weekends. The credit of a modular course is computed based on the total number of equivalent lecture hours (H). 

\begin{itemize}
	\item A modular course 14 to 20 hours carries 1 credit.
	\item A modular course with 7 to 13 hours is worth 0.5 credits.
\end{itemize}

\subsection{Programs of study}

Admissions to UG programs will be based on the Joint Entrance Examination (Advanced) conducted by the Joint Seat Allocation Authority (JoSAA) set up by the Ministry of Education. Number of seats available for any program is decided by the senate based on the norms and regulations of Government of India (GoI).

\subsubsection{Major program}

Students are currently admitted to the following 4-year Bachelor of Technology (B.Tech.) UG programs at IIT Goa.

\begin{itemize}
	\item Computer Science and Engineering (CSE)
	\item Electrical Engineering (EE)
	\item Mathematics and Computing (MnC)
	\item Mechanical Engineering (ME).
\end{itemize}

\subsubsection{Minor program}

The objective of minor is to give an opportunity to the students to go beyond their major to get an in-depth knowledge of an area outside of or complimentary to their major. The students who opt for Minors need to complete the requirements within the stipulated time of 4 years. The schools/ AUs will publish eligibility criteria, selection process, the detailed structure, syllabus, and requirements for acquiring minors. To get a minor degree, students should satisfy all criteria specified by the academic unit offering the minors program. This will include taking the designated courses, maintaining certain SPI and so on. The modalities of withdrawal from a Minors program are also specified in the minor program policy. The general description of the program is as follows: 

\begin{itemize}
	\item \textbf{General Minor} This is offered by a specific academic unit and the objective is to equip a student who is majoring in an area with the fundamentals of another discipline. As an example, a computer science engineering student has the option to earn a minor in mechanical engineering.
	\item \textbf{Concentration minor} Currently no concentration minor is offered at IIT Goa.
\end{itemize}

For more information, please consult the Minor program policy.
	
	
\end{document}