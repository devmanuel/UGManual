\section{Leave Rules and Attendance Requirements}

\subsection{Class Attendance Rules}

The institute expects 100\% attendance in every course the student has registered. If for some reason, a student is not able to attend a course, he/she is expected to inform the instructor of the course as well as the Dean of the Academic program in advance. The instructor may decide on an attendance rule and announce it at the beginning of the course. The following are some examples of attendance rules followed by the instructors.

\begin{enumerate}
    \item If the student fails to meet a minimum attendance of 80\%, he/she may be debarred from appearing in the end-semester examination and given grade X. Such a student will have to re-register for the same course if it is a core course (see grading policy).
    \item The instructor may give a certain weightage in the student’s performance evaluation of the course to attendance.
\end{enumerate}

Additionally, attendance is mandatory for the first two weeks of each semester, during which the students are allowed to adjust their course registration. A student not satisfying this criterion may be deregistered from the course.

\subsection{Vacations and Leave Rules}

Undergraduate students are entitled to avail the winter and summer vacations as specified in the academic calendar without seeking any permission. Additional leave rules are described below.

\subsubsection{Short Leave}

Leave of absence during the semester is discouraged for all registered students. However, for bona fide reasons, a student may apply for leave. The following points are considered for granting leave to the students.

\begin{enumerate}
    \item A student may be granted leave for a medical emergency. A medical certificate from the institute Health Centre or a Government Hospital (endorsed by IIT Goa) is necessary for getting leave on health/medical grounds. The extent of this leave for medical reasons can be a maximum of ten working days per regular semester. 
    \item A maximum of five working days of leave may also be granted for any other valid reason (like attending sports/cultural/technical events) in a regular semester. Personal reasons are not considered a valid reason for applying for a leave.
    \item In no case may a student be granted leave of absence in excess of fifteen working days in a regular semester. 
    \item The leave of absence in the summer term shall correspondingly be five working days (medical) and three working days (others), i.e., eight working days total. 
    \item Application for leave of absence should be submitted to the Academic Office through the Faculty Advisor well in advance.
\end{enumerate}

\subsubsection{Temporary Withdrawal/Semester Leave}

 A student may be allowed a leave of absence for a whole semester (temporary withdrawal) for bona fide reasons. Withdrawal on medical grounds/other exceptional reasons may be permitted by SUGC up to a maximum of two semesters during the students' entire program.

\begin{enumerate}
    \item An application for temporary withdrawal should be made before the date of registration for the semester as mentioned in the Academic Calendar. However, under exceptional circumstances, a student may apply for withdrawal before the start of the end semester exam.
    \item Application for temporary withdrawal should be addressed to the Dean, and routed through the Faculty Advisor. It should be submitted to the Academic office with supporting documents such as a Medical certificate (in original) in case of an illness. 
    \item A student who remains on authorized leave of absence due to ill health shall be required to submit a certificate from a Registered Medical Practitioner, ratified by the MO, IIT Goa to the effect that s/he is sufficiently cured and is fit to resume her/his studies. The registration of the student shall be provisional till the Medical Officer certifies the fitness. In the event that the MO recommends that the student is not yet fit to resume studies, the registration may be cancelled.
\end{enumerate}

The students on semester leave or temporary withdrawal should pay the continuation fee as prescribed in the fee structure.

\subsection{Penalty for Unsanctioned or Excessive Leave}

If a student is found to be absent from class without sanctioned leave, then the course instructor may recommend deregistration of the student from the course. The policy regarding unsanctioned leave leading to de-registration or any other consequence must be declared by the instructor at the beginning of the course. This rule applies to courses in regular semesters as well as in the summer term. If a student is found to be absent from a majority of lectures, tutorials and laboratory sessions for more than 20 working days (not necessarily contiguous) in a semester, with or without sanction, then her/his registration for all the courses in that semester may be cancelled by Senate on recommendation of SUGC resulting in a forced semester drop. If a student is found to be absent from all academic activities in a semester without authorization for more than 30 working days contiguously or s/he does not appear, without a compelling reason, for the end-semester examinations in all the courses in which s/he is registered, then her/his program will be terminated.