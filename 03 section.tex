

\section{Academic Sessions, Credit system, and Programs of Study}

\subsection{Semesters \label{lab:Semester}}

The academic session normally runs from the end of July in one year to the middle of July in the next year. It is typically divided into three parts described below.

\begin{enumerate}[leftmargin=15mm]
	\item \textbf{Autumn Semester}{} From the fourth week of July to the last week of November.
	\item \textbf{Spring Semester}{} From the last week of December to the last week of April.
	\item \textbf{Summer Term}{} From the middle of May to the middle of July. Spring semester is \textit{not} considered as a regular semester.
\end{enumerate}

Each of the two regular semesters consists of about seventeen to nineteen weeks. About seven to nine working days of each semester are used for the end-semester examination and one week during the semester is utilized for the mid-semester examination. The equivalent of fourteen weeks is devoted to teaching, which excludes all holidays and days spent on both the examinations, in each semester. The summer term consists of eight teaching weeks, not including holidays and examinations days.

\subsection{Academic Calendar \label{lab:Academic Calendar}}

The dates of all academic activities including those of registration, late registration, last date of document submission, first and the last days of classes, dates to add or drop courses, course feedback collection, examinations, deadline for final grade submission, mid-semester recess, and vacation are published in the academic calendar every year by the academic office.

\subsection{Credit System \label{lab:Credit System}}

The \glspl{student} admitted to the \acrshort{ug} program are expected to complete eight semesters of academic work. IIT Goa employs an academic credit system that is a standard used to quantify the efforts taken by the student towards their academic targets. Each course is assigned a certain number of credits based on the number of engagement hours of instruction. Their credits are calculated by means of the lecture hours per week (L), tutorial hours per week (T) and practical hours per week (P). Each program specifies the total number of credits required for graduation as described in a later section. The courses students can take are categorized as either standard courses or modular courses.

\paragraph{Standard courses} These courses are either full semester or half-semester long. The calculation of credits for standard courses depends on whether the course involves lecture hours. For a full semester course, when the course involves lectures, the credit is computed as shown below.

\begin{equation}
    \mathrm{Credits}=L+T+\mathit{floor}\left(\frac{P}{2}\right)
    \label{lab:CreditFloor}
\end{equation}

If the course contains no lecture hours, the credits are calculated using the formula below.

\begin{equation}
    \mathrm{Credits}=L+T+\mathit{ceiling}\left(\frac{P}{2}\right)
    \label{lab:CreditCeiling}
\end{equation}


The credit of a half-semester course is half of that of a full semester course with similar L-T-P parameters. The \Cref{tab:credits-example} exemplifies the above procedure for various combinations of L-T-P hours.

\begin{table}[t]
	
    \centering

    \begin{tabular}{c c c c c}
    	    \toprule
    \textbf{L (Lecture)} & \textbf{T (Tutorial)} & \textbf{P (Practical)} & \textbf{Credits (Full)} & \textbf{Credits (Half)}\\
    
\midrule
    
     3 & 0 & 0 & 3 & 1.5   \\
     3 & 1 & 0 & 4 & 2 \\
     3 & 0 & 3 & 4 & 2 \\
     3 & 0 & 2 & 4 & 2   \\
     0 & 0 & 3 & 2 & 1  \\
    \bottomrule
    \end{tabular}
    \caption{llustration of course credits for common L-T-P parameters}
    \label{tab:credits-example}
\end{table}


\paragraph{Modular Courses} A modular course is one that has a shorter duration than a half semester courses and is flexible in its schedule compared to a standard course. Unlike a standard course, a modular course can start any time during the semester and its scheduling could be of a different form.

For instance, a modular course could consist of engagement hours totaling to 14 hours in a span of 3 weeks or over several weekends.  

\begin{enumerate}[leftmargin=15mm]
	\item The credit of a modular course is computed based on the total number of equivalent lecture hours (H).
    \item A modular course 14 to 20 hours carries 1 credit.
    \item A modular course with 7 to 13 hours is worth 0.5 credits.
\end{enumerate}

\subsection{Programs of study}

Admissions to \acrshort{ug} programs will be based on the Joint Entrance Examination (Advanced) (see \url{https://jeeadv.ac.in/}) conducted by the Joint Seat Allocation Authority (JoSAA) (see \url{https://josaa.nic.in/}) set up by the Ministry of Education. Number of seats available for any program is decided by the \gls{senate} based on the norms and regulations of Government of India (GoI).

\paragraph{Major program} Students are currently admitted to the following 4-year \acrfull{btech} \acrshort{ug} programs at IIT Goa.

\begin{itemize}[leftmargin=15mm]
	\item Computer Science and Engineering (CSE)
	\item Electrical Engineering (EE)
	\item Mathematics and Computing (MnC)
	\item Mechanical Engineering (ME).
\end{itemize}

\paragraph{Minor program} The objective of minor is to give an opportunity to the \glspl{student} to go beyond their major to get an in-depth knowledge of an area outside of or complimentary to their major. The students who opt for Minors need to complete the requirements within the stipulated time of 4 years. The \glspl{school}/ \acrshortpl{au} will publish eligibility criteria, selection process, the detailed structure, syllabus, and requirements for acquiring minors. To get a minor degree, students should satisfy all criteria specified by the \acrshort{au} offering the minors program. This will include taking the designated courses, maintaining certain SPI (see \Cref{lab:SPICPI})and so on. The modalities of withdrawal from a Minors program are also specified in the minor program policy. The general description of the program is as follows: 

\begin{itemize}[leftmargin=15mm]
	\item \textbf{General Minor} This is offered by a specific academic unit and the objective is to equip a student who is majoring in an area with the fundamentals of another discipline. As an example, a computer science engineering student has the option to earn a minor in mechanical engineering.
	\item \textbf{Concentration Minor} Currently \textit{no} concentration minor is offered at IIT Goa.
\end{itemize}

For more information, see the Minor program policy.

\subsection{Branch Change}

Branch change decisions will be taken at the end of first two semesters based on second semester performance. Additional criteria and evaluations may be put in place through the branch change policy.