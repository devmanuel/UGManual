\section{Curriculum}

Every \acrshort{btech} program has a prescribed course structure, termed as curriculum whose details are available on the Institute's website or through the \acrshort{pugc} (see \Cref{lab:PUGC}) of the \gls{program}. The courses of study may be updated every semester and are available on the \gls{institute}'s website. The requirements for degree program run by the \Gls{institute} are broadly classified into institute requirements and program requirements. The later is further divided into compulsory or core courses, elective courses, and other requirements (projects, internship and seminars). 

\subsection{Institute Core Component}

Courses that are part of institute core are common and compulsory to all \acrshort{ug} students. These courses are offered in the first two semesters (see \Cref{lab:Semester}). The constitution of the institute core component is summarized in \Cref{tab:first semester courses}, \Cref{tab:second semester courses} and \Cref{tab:Credits for institute core}.

\begin{table}[b!]
    \centering
    \begin{tabular}{c c c c}
        \toprule
        \textbf{Course Code} &  \textbf{Course Title} & \textbf{L-T-P} & \textbf{Credits} \\
        \midrule
        MA101   & Mathematics 1                 & 3-1-0 & 4 \\
        PH101   & Physics 1                     & 2-1-2 & 3 \\
        CS101   & Introduction to Computing     & 3-0-2 & 4 \\
        HS101   & Foundation Program in HSS     & 3-0-0 & 3 \\
        CH101   & Chemistry                     & 3-1-0 & 4 \\
        CH102   & Chemistry Lab                 & 0-0-3 & 2 \\
        ME101   & Introduction to Manufacturing & 0-0-3 & 2 \\
        XX100   & Introduction to Profession    & 1-0-0 & 0 \\
        \midrule
        {}      & {}                            & Total & \textbf{22} \\
        \bottomrule
    \end{tabular}
    
    \caption{The theory courses and laboratory courses for first year students in their \textit{first semester}}
    \label{tab:first semester courses}
    
\end{table}

\begin{table}[t]
    \centering
    \begin{tabular}{c c c c}
        \toprule
        \textbf{Course Code} &  \textbf{Course Title} & \textbf{L-T-P} & \textbf{Credits} \\
        \midrule
        MA102   & Mathematics 1                             & 3-1-0 & 4 \\
        PH102   & Physics 1                                 & 2-1-0 & 3 \\
        PH103   & Physics Lab                               & 0-0-3 & 4 \\
        BIO101  & Introduction to Biology                   & 3-0-0 & 3 \\
        ME102   & Introduction to Computer Aided Graphics   & 1-0-3 & 4 \\
        EE101   & Introduction to Electrical Engineering    & 3-0-3 & 2 \\
        XX101   & Program Core I                            & 3-0-Z & 3 + floor (z/2) \\
        \midrule
        {}      & {}                                        & Total & \textbf{22 + floor (z/2)} \\
        \bottomrule
    \end{tabular}

    \caption{The theory courses and laboratory courses for first year students in their \textit{second semester}}
    \label{tab:second semester courses}
    
\end{table}

\begin{table}[b!]
    \centering
    \begin{tabular}{c c}
        \toprule
        \textbf{Stream}                     & \textbf{Credits} \\
        \midrule
        Mathematics                         & 4 \\
        Physics                             & 3 \\
        Chemistry                           & 4 \\
        Biology                             & 3 \\
        Mechanical Engineering              & 4 \\
        Electrical Engineering              & 2 \\
        Computer Science and Engineering    & 2 \\
        Humanities and Social Sciences      & 3\\
        \midrule
        Total & \textbf{40} \\
        \bottomrule
    \end{tabular}

    \caption{Credits for institute core}
    \label{tab:Credits for institute core}
    
\end{table}

\subsection{Program Core Component}

The program core component constitutes at least 60 credits and these courses are mandatory. Semester wise courses offered as part of the program core component for the \acrshort{btech} program are available on the website.

\subsection{Humanities and Social Sciences Component}

The \acrshort{ug} program will have at least 12 credits from courses offered by School of Humanities and Social Sciences (SHSS), of which 3 credits are obtained in the first semester as a compulsory institute core component as listed above. The remaining 9 credits can be obtained in a distributed manner from any open elective courses offered by the SHSS. 

\subsection{Elective Component}

\paragraph{Program Electives} These are elective courses that students take from their parent discipline. Students typically have to take at least 12 credits as program electives.

\paragraph{Open Electives} These are the elective courses that students may take from any \acrshort{au}/\gls{program} in the Institute who have declared a course to be an open elective. They have to take at least 21 credits from open electives out of which 9 credits are included in the HSS component. The list of available open electives is provided by the academic office.

\subsection{B.Tech. Project and Internship}

\glspl{student} may have to earn additional credits from a combination of B.Tech. project and program electives or a six-month internship. The exact details may vary from program to program. The general description of each is given below.

\paragraph{B.Tech. Project (BTP)} The BTP is a course wherein a final year \gls{student} participates in research and development work under the guidance of a \gls{faculty} and thus be initiated into the methods of research, library reference work, use of engineering scientific equipment/instruments, learning of modern computational techniques, and writing of technical and scientific reports. BTP involves the application of knowledge earned while undergoing various courses and labs. During the project, the student may have to conduct a literature survey, carry out theoretical analysis, conduct an experimental investigation, design a prototype, analyze data, fabricate new equipment, or any combination of these elements.

The B.Tech. project is carried out in the final year and may be divided into two stages with the first stage carried out in the Autumn Semester and the second stage in the subsequent Spring semester (see \Cref{lab:Semester}). The credit details (see \Cref{lab:Credit System}), BTP rules, prerequisites and other parameters of the BTP are defined by the program. However, typical BTP are structured with following pointers; 

\begin {enumerate}[leftmargin=15mm]
    \item \Glspl{student} have to take a project under the guidance of a \gls{faculty} from the same program unless specifically permitted by the \gls{program}. The allotment of project advisors should to be completed within two weeks from the date of commencement of semester.
    \item The assessment of the project work typically consists of a presentation and/or a report for both mid-semester and end-semester evaluations The \acrshortpl{au} may specify additional requirements, rubrics, and procedures. The final project evaluation must be completed along with the theory courses. 
    \item The guide will ensure that the work carried out by the \gls{student} is adequate, before giving approval for submission of the project report for evaluation. 
    \item The guide may award grade ‘X' (see grading policy) (at least 1 month in advance of stage evaluation) in case the \gls{student} has been irregular in interactions and work (a \gls{student} is typically expected to have at least one meeting every week with the guide). 
    \item In special situations (such as prolonged health problems, delay in getting facilities), the guide may recommend to \acrshort{sugc} (through \acrshort{pugc}) an extension of a maximum of one month for submission and evaluation without any grade penalty. 
    \item \label{lab:clause 4.5.6} The examination panel may award an incomplete grade (IN) (see grading policy) for poor performance and/or inadequate work. In the case of grade ‘IN’, the \gls{student} can appear again after one month by submitting a fresh report; in this case, the maximum possible grade will be restricted to C as in the case of courses (see grading policy).
    \item The \gls{student} will be required to register afresh for the stage in case of FX and F grades. In case of grade F, the summer registration may be permitted by \acrshort{sugc} on a case-to-case basis and on justified recommendation of \acrshort{pugc}. However, registration during summer is \textit{not} permissible in case of FX grade. 
    \item \label{lab:clause 4.5.8} A grade ‘IN’ may also be awarded if the student misses the evaluation on medical grounds, in which case, a re-examination must be held before the start of next semester. Grade ‘IN’ grade will be converted to ‘F’ grade if the \glspl{student} fail to complete the requirements within the stipulated time. The \gls{student} needs to submit a medical certificate endorsed by the IIT Goa medical officer.
    \item In case of delayed project submission other than those in 4.5.\ref{lab:clause 4.5.6} and \ref{lab:clause 4.5.8} above, the maximum permissible grade is ‘B’.
\end{enumerate}

\paragraph{Six Months Internship} In order to enhance industry-academic partnerships and to improve placement opportunities, IIT Goa has introduced the possibility of six-month internships in the B.Tech. curriculum. \Glspl{student} during their 7\textsuperscript(th) semester (tentatively middle of May to middle of November) are available to take of internships in an industry or in an academic institution. Those who are not interested in an external internship may work on an academic project at IIT Goa. In order to ensure the scientific quality of internships and to facilitate uniform grading, an internship advisory committee will be formed for each \acrshort{au}.

During the selection process itself, the companies are expected to provide details of the project as well as the duties of the \gls{student} so that the committee can evaluate the quality of the proposed work. Based on the above criterion, the \gls{program}/\gls{school} level committee can approve or reject internship positions. \glspl{student} are allowed to take only those internships that are approved by the \gls{program}/\gls{school}. After selection, a guide from both IIT Goa and in the industry will be allotted. The \gls{faculty} guide will interact with the industry counterparts for continuously monitoring the progress. At the end of the internship period, each \gls{student} must submit a project report, and make a presentation to the \gls{program}/\gls{school}-specific evaluation committee. Grades will be given based on the quality report and presentation as well as on the feedback from the supervisor from the industry. The exact modalities of evaluation and grading will be decided by the \gls{program}/\gls{school}.

\subsection{National Sports Organization (non-credit course)}

\Glspl{student} must register for any one of the activities under National Sports Organization (NSO) during the first two semesters (see \Cref{lab:Semester} as a mandatory requirement. The choice will be based on the aptitude of the \gls{student} for any of these activities and will be decided by a committee constituted for the purpose. Formal registration (see \Cref{lab:Registration} must be done for any one of these activities only at the beginning of the first two semesters along with other courses. Once registered for one of these activities, no change will be permitted at a later stage. This requirement must be completed before the end of the second year. In case valid reasons exist, a student may be given special permission for completion of this requirement before the end of the third year failing which they will not be permitted to register for the courses of the fourth year. A \gls{student} will be awarded grade P (Pass) for this activity in each semester provided the minimum requirement of this activity is met during that semester together with 80\% minimum attendance failing which the grade F (Not Pass) will be given. The award of the degree (see \Cref{lab:Award of Degree}) is subject to the successful completion of NSO. In addition to the above mandatory requirements, the students are also permitted to take NSO in subsequent years. This facility is specially meant for students having interest in NSO activities.