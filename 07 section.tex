\section{Performance Assessments}

Semester wise performance assessment of every registered student is done through various modes of examinations. These include quizzes, class tests, home assignments, group assignments, viva voce, Mid Semester and End Semester Examinations. The Instructor will announce the modes of evaluation and distribution of weightage for each of the assessments at the beginning of the course. Various modes of assessment for theory and laboratory courses along with the recommended relative weightage of various components, are given in this section. A large departure from the recommended modes of assessments and weightage will require prior approval from the PUGC and SUGC. This section can be a pointer for typical practices f weights of the evaluation components, rather than being a rule. Instructors will be free to choose weightages of exams, quizzes, assignments, class interactions, etc. and adapt a system where assessment is done in a 'continuous' manner rather than through one or two examinations.

\subsection{Modes of Evaluation}

\subsubsection{Theory Courses}

\begin{enumerate}
    \item Relative weightage for in semester evaluations including mid-semester examination is typically between 50 and 75 per cent. This will consist of one mid semester test of two hours’ duration, of about 15 to 30 per cent weightage, to be held as per the schedule fixed in the Academic Calendar. The Instructor may also set aside up to a maximum of 10 per cent of the in semester marks for active participation in the class and the initiatives shown by the student. Makeup for any absence from in semester evaluations like mid sem/tests/quizzes will be at the discretion of the Instructor. Instructor needs to be convinced that the reasons for absence are genuine. 
    \item The end-semester examination is mandatory and will be held as per the Academic Calendar and the relative weightage for this would be 25 to 50 per cent. It is normally of 3 hours’ duration and typically covers the full syllabus of the course. The end-semester examination is mandatory. 
    \item For the half semester courses, the final evaluation will be conducted along with the Mid Semester examinations of the full semester courses.
    \item For modular courses, all the evaluation components should be completed within 1-2 weeks of the completion of the course. The evaluation policy must be stated clearly in the course proposal and approved by SUGC. The attendance alone cannot be considered for evaluation.
\end{enumerate}

\subsubsection{Laboratory Courses}

The assessment in a laboratory course will be based on periodic evaluation of the student’s work prescribed through laboratory journals, their performance in viva voce examinations and group discussions, and an end-semester examination which may constitute a practical exam, a written exam, or a project. In-semester work typically carries 60-75\% and the end-semester test 40-25\% weightage respectively. It is obligatory to maintain a laboratory journal as prescribed by the course instructor. Final examination for laboratory courses will normally be held a week before the final theory examinations. In case the student remains absent from the laboratory end-semester examination, the same rules as those for theory courses are applicable.

\subsubsection{B.Tech. Project and Six-month Internship}

The students not opting for the 6-month internship in the 7th semester may have to enroll for the B.Tech. project part I (BTP-I) which may be compulsory depending on the program requirements. B.Tech. project part II (BTP-II) is an elective and BTP-I may be the prerequisite for BTP-II. Program may prescribe a minimum performance in BTP-I (which may be higher than the pass grade ‘D’ (see grading policy)) for being eligible for BTP II. BTP I and BTP- II are separately graded, at the end of the respective semesters. 
BTP-I, BTP-II, and the six-month internship are supervised, and need regular interaction (at least once a week) with the supervisor. Students may have to submit a project report and defend it in front of a panel of examiners consisting of the advisors and at least two additional faculty. The dates for submission of reports, the dates for presentations, and details of mode of assessment are decided by the individual programs. The grades must be sent to the Academic office before the last date for submission of grades. In case of irregularity of interactions, the supervisor may award grade “F” (see grading policy) before the last date for submission of the report.Grade “IN” (see grading policy) may be awarded if the student misses the presentation.

\subsection{Grading and Performance Indices}

\subsubsection{Grading Policy}

IIT Goa grading policy defines the standard letter grades throughout the institute for all courses. The assignment of a course grade is solely on the basis of the student performance in the course. Instructors are free to set the grading criteria in their courses and consider different components such as class quizzes, mid-semester/end-semester exams, assignments, projects, classroom attendance, etc., in the performance evaluation.  A separate grading policy provides the details of various grades awarded at IIT Goa.

\subsubsection{Semester Performance Index (SPI) and Cumulative Performance Index (CPI)}

The academic performance of a student in a semester is measured by Semester Performance Index (SPI). SPI is the weighted average of the final grade points obtained in all the courses taken by the student during the semester. SPI is calculated up to two decimal places by rounding up the third decimal. An example SPI calculation is as follows. Suppose in a given semester, a student has taken five courses having credits C1, C2, C3, C4, C5 and the student’s final grade points in those courses are G1, G2, G3, G4, G5 respectively. Then, the student’s SPI is

\[SPI=\frac{(C1×G1)+(C2×G2)+(C3×G3)+(C4×G4)+(C5×G5)}{C1+C2+C3+C4+C5}\]

Cumulative Performance Index (CPI) denotes the up-to-date academic performance of a student. CPI is calculated as the weighted average of the final grade points obtained in all the courses taken by the student after the first semester. The grades obtained by the students in the very first semester do not contribute to the overall CPI. The CPI calculation is done using the same formula as the SPI calculation. Fail grades are included in the calculation of SPI and CPI. The failed courses, which the student passes later, will be considered to recalculate the CPI. The SPI will not be recalculated. SPI is not calculated for a dropped semester as the semester is treated as unregistered. 

\subsubsection{Disclosure of Performance Assessment}

In semester performance of all students is communicated by the instructor to the students before the end semester examination. Those awarded grade “X” (see grading policy) as described above will be clearly identified in this list.

\begin{enumerate}
    \item A course instructor should allow students to see the evaluated answer scripts at their discretion, as long as this is done before the finalization of grades. The final total marks for an individual student should be shared with that student. The limit fixed for such a disclosure is before the date for grade submission. The time and venue will be decided and conveyed by the course Instructor.
    \item Any change in grade requires approval of the Chairman Senate.
    \item Once grades are published, changes if any will be allowed in case of totaling and tabulation errors only. 
    \item Evaluated answer scripts should be preserved by the Instructor/ Departmental Office for a minimum period of one semester. 
\end{enumerate}

\subsection{Policy on Academic Progression}

All undergraduate degree programs at IIT Goa have a program-specific curriculum that provides the course structure and the credit requirements for graduation. Each course in the curriculum concludes with a rigorous performance assessment and award of a letter grade/ grade point to the student. Subsequent calculation of SPI and CPI allows quantification of the student proficiency in the given course and program. The progression of the student in the degree program is linked with these quantifiable metrics to allow matching of their academic load with their needs and abilities. Based on the student performance, their academic standing and allowable academic loading for each semester is determined. 

IIT Goa is an institute of national importance and the students at IIT Goa are expected to excel in their studies as captured by their evaluations. It is possible that some students may be unable to cope with their studies repeatedly despite their best effort. Early recognition of such cases will permit prompt corrective action such as putting the student on a slow track of learning. In extreme cases, where the student fails to improve repeatedly, early termination may also be recommended. The provisions of this policy may be adjusted through the faculty advisor with approval from PUGC, SUGC, and competent authority only in case of genuine medical reasons.

\subsubsection{Terminology}

\begin{enumerate}
    \item \textbf{Semester Academic Load} The academic load of each semester is quantified by the number of credits prescribed by respective curricula against that semester. This number varies from semester to semester and program to program.
    \item \textbf{Available Course} A course is said to be available to a student when they are eligible to register for the course in the given semester.
    \item \textbf{Backlog Course} A backlog course is a course that satisfies one of the following criteria
    
    \begin{enumerate}
        \item A course which the student has to repeat (in case of core courses) or replace (in case of electives) due to an E/F/FX/FN grade in the previous attempt. 
        \item A mandatory course the student could not register in the previous semesters because he/she had not yet cleared prerequisite course(s).
        \item A mandatory course the student was not able to register or complete due withdrawal from the semester with due approval from the competent authorities.
        \item A mandatory course the student had to forgo due to the mandatory reduction in academic load due to poor performance in the preceding semesters.
    \end{enumerate}
    
    \item \textbf{Outstanding and Available Backlog Courses }Typically, each course is either available for registration in autumn or spring semester. Outstanding and available backlog courses are the courses in which a student has a backlog and are available for registration in the given semester.
\end{enumerate}

\subsubsection{Academic Progression Modalities}

\begin{enumerate}
    \item All students have to mandatorily register for all the prescribed courses of each semester except if they are on probation/ the slow-track of learning. 
    \item Students with backlogs have to mandatorily register for the outstanding and available backlog courses when they are available. These students may further register courses from the current semester to bring the total registered credits for the semester up to their respective semester academic load.
    \item Students may overload additional courses as per the provisions specified in Table 4. No overload is allowed in the first and second semesters. The facility for overloading is provided for allowing students to clear backlogs, take additional electives, and minor program. It cannot be used for core courses prescribed in the curriculum. The overloading of any course, even if allowed by the policy, must also be possible without a timetable clash.

    \begin{table}[h!]
        \centering
        \begin{tabular}{c c}
        \toprule
        \textbf{Criteria}   & \textbf{Maximum Academic Overload} \\
        \midrule
        $CPI \geq 7$          & Overload up to 8 credits \\
        $5 \leq CPI \le 7$    & Overload up to 4 credits \\
        $CPI < 5$             & No overloading allowed \\
       \bottomrule
        \end{tabular}
        \caption{Mapping CPI to maximum allowable academic load}
        \label{tab:Mapping CPI to maximum allowable academic load}
    \end{table}

    \item From 7\textsuperscript{th} semester onwards, the students nearing completion of the degree may be allowed to overload courses above the credit limit imposed by Table 4 to complete their graduation requirement with a maximum credit limit of 24 with approval from PUGC and SUGC. 
\end{enumerate}

\subsubsection{Academic Probation and Termination Modalities}

The students of IIT Goa are always expected to maintain at least a minimum level of performance. Failing to which, they may be put on academic probation.

\begin{enumerate}
    \item Students who are not able to earn at least 50\% of the semester credits during the 1\textsuperscript{st} OR 2\textsuperscript{nd} semester will be put on academic probation at the beginning of 3\textsuperscript{rd} semester.
    \item 3\textsuperscript{rd} semester onwards the academic performance of each UG student will be reviewed by the academic office at the end of each regular semester. The academic probation of these students will be according to the rules below.

    \begin{enumerate}
        \item Let $P$ be the total number of credits the student required to be completed till the last regular semester starting from the first semester, $Q$ be the total number of credits the student has earned in the preceding semesters, $R$ be the total number of credits the student has registered in the last semester and $S$ be the total number of credits the student has earned in the last semester.
        \item A student will be placed on an Academic Probation/Warning in the next semester if either or both of the following conditions are satisfied.

        \[Condition-I : Q \leq \frac{2}{3}P\]
        \[Condition-II : S\leq \frac{1}{2}R\]
        
    \end{enumerate}

    \item Students on academic probation are mandatorily put on a slow track of learning. These students may register for up to 12 credits only with priority given to available backlog courses. The  $P$, $Q$, $R$ and $S$ parameters for the students will be adjusted accordingly.
    \item To complete the registration process, a student on Academic Probation is required to sign an undertaking, with a countersign/ acknowledgment from the parents/guardians, incorporating the following conditions. 

    \begin{enumerate}
        \item She/he shall register with higher priority for those courses (or their substitute) in which a failed grade is obtained.
        \item She/he shall not hold any office in the Hall of Residence, Students Gymkhana or any other organization/body.
        \item She/he may face early termination if the academic performance is not improved.
        \item Any other terms and conditions laid down by the SUGC/Senate.
    \end{enumerate}

    \item Students who are found to be unable to cope with even the slow track of learning may become eligible for termination from the program. If a student already on Academic Probation/Warning satisfies either or both of the abovementioned conditions at the end of the current semester, then Academic Program of the student will be terminated.
\end{enumerate}

Consult the Guidelines for multiple entry and exit for more information on the exit policies of the Institute.

\subsubsection{Maximum Period for Completion of Program}

A student is required fulfil the requirements for their respective degree within the maximum period specified for the program, including withdrawal in exceptional circumstances, failing which their case will be referred to the Senate for dismissal. If a student fails to complete their academic. program within 12 semesters (6 years), their academic program will be automatically terminated. However, if there are valid reasons, an appeal can be submitted to the Chair, Senate for reinstatement of their academic program. If the appeal is approved, the student will be required to pay appropriate fees as mentioned in the Institute's fee structure.

\subsubsection{Appeal for Reinstatement}

A student whose program is terminated may appeal to the Chairman, Senate, for reinstatement in the program. In case of termination due to inadequate academic performance, the student should clearly explain reason(s) for poor performance, including how those reason(s) will not adversely affect her/his performance in future. The Senate shall take a final decision after considering all available inputs. A student may re-appeal even after a previous appeal has been rejected. However, the Senate may not entertain any re-appeal for review unless substantial additional information is brought to its notice.

\subsection{No-Fail Policy for First Year B.Tech. Students}

One problem that faces every one of all the IITs is the unhealthy level of stress felt by many students, often leading to severe psychiatric disorders. As new students enter an IIT, the students perceive an intense competition, in an environment entirely new to his/her experience. Further, his/her performance in the very first semester, if poor, often leads the student to identify himself/herself as a deficient student, and this early identification of oneself may lead to a sense of ‘giving up’ for the rest of the students stay in the Institute.

In view of the above to relieve the stress of the students a No-Fail policy has been approved by the senate of IIT Goa. This No-Fail policy will not reduce any Academic credit requirement for obtaining the degree. It will only relieve the stress of the students. Following are the salient features of no-fail policy:

\begin{enumerate}
    \item For the first semester will show grades only for the courses which have been cleared by the student.
    \item A course in which the student has obtained a failure grade will be treated as a course which was not registered for;
    \item The student has to pass in the failed course in the subsequent semesters for completing the requirements for obtaining the degree.
    \item Grades obtained in the First Semester by the student will appear on his/her transcript, however, No SPI (Semester Performance Index) will be computed for the first semester and the overall performance, i.e., the cumulative performance index (CPI) will be computed on the basis of the rest of the seven semesters.
    \item Rules for Branch Change with regard to the above mentioned policy are mentioned in the branch change policy.
\end{enumerate}

\subsection{Academic Malpractice}

The Institute has an approved policy regarding incidences of academic malpractice and follows it very strictly. Some of the acts can be impersonation, forging, copying in exams, assignments, lab projects etc., communicating with other students during exams, possession of chits, using electronic gadgets, plagiarism, and indiscipline by student during the exam. Disciplinary actions against academic malpractices include suspension for a semester/year, zero marking, grade degradation, no stipend/benefits, and so on. The instructors and exam invigilators are equipped with detailed instructions to be followed upon discovery of such acts. They are reported to the SSAC along with details like roll no, name, examination details, description of the act, material evidence, response of the student etc. Please check the policy document for detailed information. 