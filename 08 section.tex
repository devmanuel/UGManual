\section{Completion of Program and Graduation}

\subsection{Award of Degree \label{lab:Award of Degree}}

On successful completion of the prescribed requirements for a program, the degree will be conferred on a student in an annual convocation of the Institute. The degree certificate will indicate the relevant branch, and specializations if any, in the engineering or science discipline in which the student has graduated. Although CPI will be given in the Semester grade reports and transcript the final degree certificate will not mention any class whatsoever. In those cases, where the student has earned the required credits for a Minor in another discipline, this will be mentioned in the degree certificate.

However, for conferment of degree, student has to fulfill the following requirements in addition to the conditions described in the subsections below. 

\begin{enumerate}[leftmargin=15mm]
    \item The student should have taken and passed all the prescribed courses as per the institute and program requirements. 
    \item The student should have satisfactorily fulfilled other academic requirements such as NSO, seminar, and the B.Tech. project. 
    \item The student should have paid all the Institute dues 
    \item The student should have no case of indiscipline against him/her.
\end{enumerate}

\paragraph{Minimum Credit Requirement} The minimum credits required for graduating with a B.Tech. degree are defined by each program independently. Typically, for the award of a U.G. degree the minimum credit requirement is between 132 and 144, while the maximum credit for any B.Tech. program may not exceed 160. The overall credit requirement is further divided into credit requirement for basic science courses, HSS courses, open electives, and program courses. The detailed curriculum showing the distribution of credits under each category as well as the sequence in which the courses must be taken will be specified by the program.

\paragraph{Minimum CPI Requirement} There shall be no minimum CPI requirement for the award of the B. Tech, but the eligibility for award of degrees to the students having CPI less than 5.00 would have to be approved by the Chairman, Senate and such cases will be reported to the Senate. 

\subsection{Issue of Transcript}

Transcript is the consolidated statement of the Academic Performance of a student for all the semesters since joining the program and is given to a student on successful completion of the program along with the degree certificate.
The transcript will show all the courses a student has registered for in a given semester irrespective of the student passing or failing in the course. In case a student requires multiple attempts to pass a course, it will be shown in all semesters where the student registered for the course.

The transcript will show the SPI and overall CPI based on all the courses taken by the student. Additional courses will be shown separately, indicating also the minor, if any, earned by the student. Additional copies of the transcript can be obtained if needed, on request and upon payment of applicable fee. Students who have not yet completed the program can obtain an Interim Transcript, if needed, on request and upon payment of applicable fee.

\subsection{Award of Medals}

Various medals are awarded to the outstanding students from amongst the graduates who receive their degree at the annual convocation of the Institute. The President of India medal / The Institute Gold Medal is awarded to the most outstanding student in the Undergraduate program (B.Tech.) admitted through JEE as per the conditions and procedures as approved by the Senate from time to time. The Institute Silver Medal is awarded to the most outstanding graduate in each branch of Engineering. Only such students who have completed the program without, dropping or failing in any of the minimum credit carrying course are considered eligible for the award of medals.